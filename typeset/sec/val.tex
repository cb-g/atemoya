The tools of valuation empower the investor to judge stocks in a manner that is detached from behavioral biases and therefore from market sentiment. Valuation emphasizes companies' fundamentals and financial health.

By relying on valuation -- regardless of an active or passive investing style -- we reject the Efficient Market Hypothesis (EMH) in at least its strong and semi-strong forms.

\subsection{DCF (Deterministic)}

\subsubsection{Discounted Cash Flow Foundation}

Let us derive a stock's value per share based on discounted free cash flow and regard this value as intrinsic.

Interest, e.g., on investment streams, compounds to form the future value:
\[
\text{FutureValue}_n = \sum_{t=0}^{n} \text{InvestedCash}_t \cdot (1 + \text{InterestRate})^{n - t}
\]

To calculate the present value, e.g. of a business, we undo the compounding effect. The intrinsic value of a financial asset is the discounted value of its future cash flows:
\[
\text{PresentValue} = \sum_{t=1}^{n} \frac{\text{CashFlow}_t}{(1 + \text{DiscountRate})^t}
\]

One can go the free-cash-flow-to-equity (FCFE) and/or the free-cash-flow-to-firm (FCFF) route. It will depend on the company's circumstances to which to give modeling precedence. FCFE should lead to a more insightful intrinsic value per share (IVPS), if the debt policy is known and reliable, and FCFF should do so, if the debt structure is uncertain or complex.

\subsubsection{Cash Flow Mechanics}

\paragraph{Free Cash Flow to Equity}

$$
\text{FCFE} = \text{NI} - (1 - \text{TDR}) \cdot (\text{CAPX} - \text{D} - \Delta \text{WC})
$$

\begin{itemize}
  \item[] $\text{FCFE}$: free cash flow to equity
  \item[] $\text{NI}$: net income
  \item[] $\text{TDR}$: target debt ratio
  \item[] $\text{CAPX}$: capital expenditure
  \item[] $\text{D}$: depreciation \& amortization
  \item[] $\Delta \text{WC} = \Delta \text{CA} - \Delta \text{CL}$: change in working capital
  \item[] $\text{CA}$: current assets
  \item[] $\text{CL}$: current liabilities
\end{itemize}

\paragraph{Free Cash Flow to Firm}

$$
\text{FCFF} = \text{EBIT} \cdot (1 - \text{CTR}) + \text{D} - \text{CAPX} - \Delta \text{WC}
$$

\begin{itemize}
  \item[] $\text{FCFF}$: free cash flow to firm
  \item[] $\text{EBIT}$: earnings before interest and taxes
  \item[] $\text{CTR}$: corporate tax rate
\end{itemize}

\subsubsection{Cost of Capital}

\paragraph{Levered Beta}

$$
\beta_{\text{L}} = \beta_{\text{U}} \cdot \left(1 + (1 - \text{CTR}) \cdot \frac{\text{MVB}}{\text{MVE}}\right)
$$

\begin{itemize}
  \item[] $\beta_{\text{L}}$: levered beta
  \item[] $\beta_{\text{U}}$: unlevered beta
  \item[] $\text{MVB}$: market value of debt
  \item[] $\text{MVE}$: market value of equity
\end{itemize}

\paragraph{Cost of Equity (CAPM)}

$$
\text{CE} = \text{RFR} + \beta_{\text{L}} \cdot \text{ERP}
$$

\begin{itemize}
  \item[] $\text{CE}$: cost of equity
  \item[] $\text{RFR}$: risk-free rate
  \item[] $\text{ERP}$: equity risk premium
\end{itemize}

\paragraph{Currency-Adjusted Equity Risk Premium}

$$
\text{ERPT} = (1 + \text{ERPS}) \cdot \left( \frac{1 + \text{IT}}{1 + \text{IS}} \right) - 1
$$

\begin{itemize}
  \item[] $\text{ERPT}$: equity risk premium of target currency
  \item[] $\text{ERPS}$: equity risk premium of source currency
  \item[] $\text{IT}$: inflation of target
  \item[] $\text{IS}$: inflation of source
\end{itemize}

\paragraph{Weighted Average Cost of Capital}

$$
\text{WACC} = \frac{\text{MVE}}{\text{MVE} + \text{MVB}} \cdot \text{CE} + \frac{\text{MVB}}{\text{MVE} + \text{MVB}} \cdot \text{CB} \cdot (1 - \text{CTR})
$$

\begin{itemize}
  \item[] $\text{WACC}$: weighted average cost of capital
  \item[] $\text{CB}$: pre-tax cost of debt
\end{itemize}

\subsubsection{Growth Rate Estimation}

\paragraph{FCFE Growth Rate}

$$
\begin{aligned}
\text{FCFEGR} &= \text{ROE} \cdot \text{FCFERR} \\
&= \left( \frac{\text{NI}}{\text{BVE}} \right) \cdot \left( 1 - \frac{\text{DP}}{\text{NI}} \right)
\end{aligned}
$$

\begin{itemize}
  \item[] $\text{FCFEGR}$: FCFE growth rate estimate
  \item[] $\text{ROE}$: return on equity
  \item[] $\text{FCFERR}$: FCFE retention ratio
  \item[] $\text{BVE}$: book value of equity
  \item[] $\text{DP}$: dividends paid
\end{itemize}

\paragraph{FCFF Growth Rate}

$$
\begin{aligned}
\text{FCFFGR} &= \text{ROIC} \cdot \text{FCFFRR} \\
&= \left( \frac{\text{NOPAT}}{\text{IC}} \right) \cdot \left( \frac{\text{CAPX} - \text{D} + \Delta \text{WC}}{\text{NOPAT}} \right) \\
&= \left( \frac{\text{EBIT} \cdot (1 - \text{ETR})}{\text{IC}} \right) \cdot \left( \frac{\text{CAPX} - \text{D} + \Delta \text{WC}}{\text{EBIT} \cdot (1 - \text{ETR})} \right)
\end{aligned}
$$

\begin{itemize}
  \item[] $\text{FCFFGR}$: FCFF growth rate estimate
  \item[] $\text{ROIC}$: return on invested capital
  \item[] $\text{FCFFRR}$: FCFF reinvestment rate
  \item[] $\text{NOPAT}$: net operating profit after taxes
  \item[] $\text{IC}$: invested capital
  \item[] $\text{ETR}$: effective tax rate
\end{itemize}

\subsubsection{Present Value Calculation}

\paragraph{FCFE-Based Valuation}

$$
\begin{aligned}
\text{PVE} = \text{PVFCFE} &+ \text{PVTVFCFE} \\
= \left( \sum_{t=1}^{h} \frac{\text{FCFE}_t}{(1 + \text{CE})^t} \right) &+ \text{PVTVFCFE} \\
= \left( \sum_{t=1}^{h} \frac{\text{FCFE}_0 \cdot (1+\text{FCFEGR})^t}{(1+\text{CE})^t} \right) &+ \text{PVTVFCFE} \\
= \text{PVFCFE} &+ \left( \sum_{t=h+1}^{\infty} \frac{\text{FCFE}_t}{(1 + \text{CE})^t} \right) \\
= \text{PVFCFE} &+ \left( \sum_{t=h+1}^{\infty} \frac{\text{FCFE}_h \cdot (1+\text{TGR})^{t-h}}{(1+\text{CE})^t} \right) \\
= \text{PVFCFE} &+ \left( \sum_{t=h+1}^{\infty} \frac{\text{FCFE}_h}{(1+\text{CE})^h} \cdot \left( \frac{1+\text{TGR}}{1+\text{CE}} \right)^{t-h} \right) \\
= \text{PVFCFE} &+ \left( \frac{\text{FCFE}_h}{(1+\text{CE})^h} \sum_{k=1}^{\infty} \left( \frac{1+\text{TGR}}{1+\text{CE}} \right)^k \right) \\
\overset{\text{GGM}}{=} \text{PVFCFE} &+ \left( \frac{\text{FCFE}_h}{(1+\text{CE})^h} \cdot \frac{\frac{1+\text{TGR}}{1+\text{CE}}}{1 - \frac{1+\text{TGR}}{1+\text{CE}}} \right) \\
= \text{PVFCFE} &+ \left( \frac{\text{FCFE}_h}{(1+\text{CE})^h} \cdot \frac{1+\text{TGR}}{\text{CE} - \text{TGR}} \right) \\
= \text{PVFCFE} &+ \left( \frac{\text{FCFE}_h \cdot (1+\text{TGR})}{(\text{CE} - \text{TGR})} \cdot \frac{1}{(1+\text{CE})^h} \right) \\
= \text{PVFCFE} &+ \left(\frac{\text{FCFE}_{h+1}}{(\text{CE} - \text{TGR})} \cdot \frac{1}{(1 + \text{CE})^h} \right) \\
= \text{PVFCFE} &+ \left(\frac{\text{TVFCFE}}{(1 + \text{CE})^h} \right) \\
= \left( \sum_{t=1}^{h} \frac{\text{FCFE}_0 \cdot (1+\text{FCFEGR})^t}{(1+\text{CE})^t} \right) &+ \left(\frac{\text{FCFE}_h \cdot (1+\text{TGR})}{(\text{CE} - \text{TGR}) \cdot (1 + \text{CE})^h} \right)
\end{aligned}
$$

\begin{itemize}
  \item[] $\text{PVE}$: present value of equity
  \item[] $\text{PVFCFE}$: present value of free cash flow to equity
  \item[] $\text{PVTVFCFE}$: present value of terminal value of FCFE
  \item[] $h$: growth forecast horizon (before terminal growth rate into perpetuity)
  \item[] $\text{GGM}$: Gordon growth model (closed-form solution of infinite geometric series)
  \item[] $\text{TGR}$: terminal growth rate
  \item[] $\text{TVFCFE}$: terminal value of FCFE
\end{itemize}

\paragraph{FCFF-Based Valuation}

$$
\begin{aligned}
\text{PVF} = \text{PVFCFF} &+ \text{PVTVFCFF} \\
= \left( \sum_{t=1}^{h} \frac{\text{FCFF}_t}{(1 + \text{WACC})^t} \right) &+ \text{PVTVFCFF} \\
= \left( \sum_{t=1}^{h} \frac{\text{FCFF}_0 \cdot (1+\text{FCFFGR})^t}{(1+\text{WACC})^t} \right) &+ \text{PVTVFCFF} \\
= \text{PVFCFF} &+ \left( \sum_{t=h+1}^{\infty} \frac{\text{FCFF}_t}{(1 + \text{WACC})^t} \right) \\
= \text{PVFCFF} &+ \left( \sum_{t=h+1}^{\infty} \frac{\text{FCFF}_h \cdot (1+\text{TGR})^{t-h}}{(1+\text{WACC})^t} \right) \\
= \text{PVFCFF} &+ \left( \sum_{t=h+1}^{\infty} \frac{\text{FCFF}_h}{(1+\text{WACC})^h} \cdot \left( \frac{1+\text{TGR}}{1+\text{WACC}} \right)^{t-h} \right) \\
= \text{PVFCFF} &+ \left( \frac{\text{FCFF}_h}{(1+\text{WACC})^h} \sum_{k=1}^{\infty} \left( \frac{1+\text{TGR}}{1+\text{WACC}} \right)^k \right) \\
\overset{\text{GGM}}{=} \text{PVFCFF} &+ \left( \frac{\text{FCFF}_h}{(1+\text{WACC})^h} \cdot \frac{\frac{1+\text{TGR}}{1+\text{WACC}}}{1 - \frac{1+\text{TGR}}{1+\text{WACC}}} \right) \\
= \text{PVFCFF} &+ \left( \frac{\text{FCFF}_h}{(1+\text{WACC})^h} \cdot \frac{1+\text{TGR}}{\text{WACC} - \text{TGR}} \right) \\
= \text{PVFCFF} &+ \left( \frac{\text{FCFF}_h \cdot (1+\text{TGR})}{(\text{WACC} - \text{TGR})} \cdot \frac{1}{(1+\text{WACC})^h} \right) \\
= \text{PVFCFF} &+ \left( \frac{\text{FCFF}_{h+1}}{(\text{WACC} - \text{TGR})} \cdot \frac{1}{(1 + \text{WACC})^h} \right) \\
= \text{PVFCFF} &+ \left( \frac{\text{TVFCFF}}{(1 + \text{WACC})^h} \right) \\
= \left( \sum_{t=1}^{h} \frac{\text{FCFF}_0 \cdot (1+\text{FCFFGR})^t}{(1+\text{WACC})^t} \right) &+ \left( \frac{\text{FCFF}_h \cdot (1+\text{TGR})}{(\text{WACC} - \text{TGR}) \cdot (1 + \text{WACC})^h} \right)
\end{aligned}
$$

\begin{itemize}
  \item[] $\text{PVF}$: present value of the firm
  \item[] $\text{PVFCFF}$: present value of free cash flow to firm
  \item[] $\text{PVTVFCFF}$: present value of terminal value of FCFF
  \item[] $\text{TVFCFF}$: terminal value of FCFF
\end{itemize}

\paragraph{Reconciliation}

$$
\text{PVE} \approx \text{PVF} - \text{MVB} + \text{CNOA}
$$

\begin{itemize}
  \item[] $\text{CNOA}$: cash and non-operating assets
\end{itemize}

\paragraph{Intrinsic Value Per Share}

$$
\text{IVPS} = \frac{\text{PVE}}{\text{SO}}
$$

\begin{itemize}
  \item[] $\text{IVPS}$: intrinsic value per share
  \item[] $\text{SO}$: shares outstanding
\end{itemize}

\subsubsection{Interpretation}

Having derived an $\text{IVPS}$, it may serve as a reference point when judging the market price. Supposing that, ideally, both (rather than just one of) the FCFE and FCFF based IVPS have been modeled well and are therefore reliable indicators, we may interpret the nine resulting constellations as follows.

\renewcommand{\arraystretch}{1.8}

\noindent
\begin{tabular}{>{\raggedright\arraybackslash}p{0.15\textwidth}
                >{\raggedright\arraybackslash}p{0.28\textwidth}
                >{\raggedright\arraybackslash}p{0.28\textwidth}
                >{\raggedright\arraybackslash}p{0.28\textwidth}}

\textbf{}
& \textbf{FCFE $>$ Price}
& \textbf{FCFE $\approx$ Price}
& \textbf{FCFE $<$ Price} \\

\textbf{FCFF $>$ Price}
& \signalstrongbuy{} The firm's assets and operations generate more value than what is priced in by the market, and equity holders retain it — low leverage or efficient debt structure.
& \signalbuy{} The firm's assets and operations are underpriced, but excess value is absorbed by debt or reinvestment, leaving equity fairly valued.
& \signalcaution{} The firm's assets and operations are underpriced, but debt or reinvestment absorbs most cash flows — equity claims more than it economically receives. \\

\textbf{FCFF $\approx$ Price}
& \signalbuy{} The firm's assets and operations are fairly priced, but equity captures a disproportionately large share — market underprices the equity upside.
& \signalhold{} The present value of free cash flows — to the firm (before payments to debt holders) and to equity (after them) — is consistent with market prices; no mispricing is evident.
& \signalspeculative{} The business generates enough pre-financing cash flow to justify its market price, but equity holders retain too little after payments to debt holders. \\

\textbf{FCFF $<$ Price}
& \signalcaution{} The business is overvalued, but equity appears cheap due to temporarily favorable debt terms — value may be unstable under a leveraged structure.
& \signalspeculative{} Equity is fairly priced, but depends on cash flows from a business generating less than what its market price would suggest — any decline in operations could undermine equity value.
& \signalavoid{} There isn't sufficient cash flow to the business or the equity for the fundamentals to justify the high market price. \\
\end{tabular}

\newpage\subsection{DCF (Probabilistic)}

\subsubsection{Motivation}

The deterministic DCF model relies on point estimates for key inputs (growth rates, discount rates, tax rates). However, these parameters are inherently uncertain. Probabilistic DCF addresses this by:

\begin{enumerate}
  \item Treating financial statement line items as random variables
  \item Propagating uncertainty through the cash flow projection
  \item Generating a distribution of intrinsic values rather than a single point estimate
  \item Quantifying valuation confidence via probability metrics
\end{enumerate}

\subsubsection{Input Sampling}

\paragraph{Historical Distributions}

The following variables are sampled independently in each simulation iteration $i = 1, \ldots, N$ (typically $N = 5000$):

\begin{itemize}
  \item Net income: $\text{NI}^{(i)}$
  \item Earnings before interest and taxes: $\text{EBIT}^{(i)}$
  \item Capital expenditures: $\text{CAPX}^{(i)}$
  \item Depreciation \& amortization: $\text{D}^{(i)}$
  \item Current assets: $\text{CA}^{(i)}$
  \item Current liabilities: $\text{CL}^{(i)}$
\end{itemize}

For each variable $X$, we compute empirical mean $\mu_X$ and standard deviation $\sigma_X$ from the historical time series (typically 4--8 quarters).

\paragraph{Sampling Distributions}

We choose the sampling distribution based on the sign and scale of the variable:

\begin{itemize}
  \item If $\mu_X > 0$, the variable is sampled from a log-normal distribution to ensure positivity:
  \[
  \log\text{-mean} = \log(\max(\mu_X, 10^{-3})), \quad \log\text{-std} = \frac{\sigma_X}{\mu_X}
  \]
  \[
  X^{(i)} \sim \text{LogNormal}(\log\text{-mean}, \log\text{-std})
  \]

  \item If $\mu_X \leq 0$, the variable falls back to a normal distribution:
  \[
  X^{(i)} \sim \mathcal{N}(\mu_X, \sigma_X^2)
  \]
\end{itemize}

No additional prior smoothing is applied to these inputs. Their randomness enters directly into the cash flow projection models.

\paragraph{Outlier Squashing}

To reduce the impact of extreme outliers, selected variables like $\text{NI}$ and $\text{EBIT}$ are post-processed using a squashing function:
\[
x \mapsto
\begin{cases}
x, & x \leq \tau \\
\tau + \log(1 + (x - \tau)), & x > \tau
\end{cases}
\]
where the threshold $\tau$ is:
\[
\tau = \min\{3 \cdot \max(\text{historical values}), \, 0.3 \cdot \text{MarketCap}\}
\]

This logarithmic damping prevents unrealistic extrapolation while preserving the shape of the distribution.

\subsubsection{Growth Rate Sampling}

\paragraph{Fundamental Growth Identity}

Two growth rates are sampled in each simulation:
\begin{itemize}
  \item FCFE growth rate: $g_{\text{FCFE}}$
  \item FCFF growth rate: $g_{\text{FCFF}}$
\end{itemize}

Both follow the fundamental identity:
\[
g = \text{return} \times \text{reinvestment rate}
\]

For FCFE:
\begin{align*}
g_{\text{FCFE}} &= \text{ROE} \times (1 - \text{payout ratio}) \\
&= \frac{\text{NI}}{\text{BVE}} \times \left(1 - \frac{\text{DP}}{\text{NI}}\right)
\end{align*}

For FCFF:
\begin{align*}
g_{\text{FCFF}} &= \text{ROIC} \times \text{reinvestment rate} \\
&= \frac{\text{NOPAT}}{\text{IC}} \times \frac{\text{CAPX} - \text{D} + \Delta \text{WC}}{\text{NOPAT}}
\end{align*}

where $\text{NOPAT} = 0.75 \cdot \text{EBIT}$ (approximating $(1 - \text{CTR}) \cdot \text{EBIT}$ with a fixed $25\%$ tax rate).

\paragraph{Bayesian Smoothing with Sector Priors}

Each return and reinvestment term is blended as:
\[
\text{final} = 0.5 \cdot \text{empirical} + 0.5 \cdot \text{prior}
\]

Sector-specific priors are sampled from:
\begin{itemize}
  \item ROE and ROIC: $\text{Beta}(2, 5)$ scaled to $[0, 0.4]$
  \item Retention and reinvestment rates: $\text{Beta}(2, 2)$ scaled to $[0, 1]$
\end{itemize}

This regularization prevents overfitting to recent historical data while incorporating industry-level information.

\paragraph{Growth Rate Distribution}

The blended product gives a per-year estimate. Growth rates are then sampled from a normal distribution using the historical mean and standard deviation:
\[
g^{(i)} \sim \mathcal{N}(\mu_g, \sigma_g^2)
\]

and clipped to user-defined bounds:
\[
g^{(i)} \in [g_{\min}, g_{\max}]
\]

Typical bounds are $g_{\min} = -0.05$ (allowing slight contraction) and $g_{\max} = 0.15$ (capping growth at 15\%).

This ensures simulated growth is consistent with both financial fundamentals and prior regularization.

\subsubsection{Stochastic Discount Rates}

In addition to stochastic growth rates, the discount rate components (risk-free rate, beta, and equity risk premium) may also be sampled to capture uncertainty in the cost of capital.

\paragraph{Discount Rate Component Sampling}

For each simulation $i$, the following components are sampled from normal distributions:

\begin{itemize}
  \item Risk-free rate:
  \[
  \text{RFR}^{(i)} \sim \mathcal{N}(\text{RFR}_{\text{base}}, \sigma_{\text{RFR}}^2)
  \]
  with typical volatility $\sigma_{\text{RFR}} = 0.005$ (50 basis points).

  \item Leveraged beta:
  \[
  \beta_{\text{L}}^{(i)} \sim \mathcal{N}(\beta_{\text{L,base}}, \sigma_{\beta}^2)
  \]
  with typical volatility $\sigma_{\beta} = 0.10$.

  \item Equity risk premium:
  \[
  \text{ERP}^{(i)} \sim \mathcal{N}(\text{ERP}_{\text{base}}, \sigma_{\text{ERP}}^2)
  \]
  with typical volatility $\sigma_{\text{ERP}} = 0.01$ (100 basis points).
\end{itemize}

\paragraph{Cost of Capital Recomputation}

Using the sampled components, the cost of equity for simulation $i$ is:
\[
\text{CE}^{(i)} = \text{RFR}^{(i)} + \beta_{\text{L}}^{(i)} \cdot \text{ERP}^{(i)}
\]

Similarly, the weighted average cost of capital is recomputed:
\[
\text{WACC}^{(i)} = \frac{\text{MVE}}{\text{MVE} + \text{MVB}} \cdot \text{CE}^{(i)} + \frac{\text{MVB}}{\text{MVE} + \text{MVB}} \cdot \text{CB} \cdot (1 - \text{CTR})
\]

This stochastic treatment of discount rates significantly increases the dispersion of the intrinsic value distribution, providing a more realistic quantification of valuation uncertainty. The feature is controlled by the \texttt{use\_stochastic\_discount\_rates} configuration flag.

\subsubsection{Cash Flow Projection}

\paragraph{Explicit Forecast Period}

For each simulation $i$, cash flows are projected over $h$ years (typically $h = 5$) using the sampled inputs and growth rate.

\paragraph{Time-Varying Growth Rates}

Rather than applying a constant growth rate throughout the projection period, the model implements exponential mean reversion toward the terminal growth rate. This prevents unrealistic perpetual high-growth assumptions.

For each simulation $i$ and forecast year $t \in \{1, \ldots, h\}$, the growth rate decays according to:
\[
g_t^{(i)} = g_{\text{terminal}} + \left(g_0^{(i)} - g_{\text{terminal}}\right) \cdot e^{-\lambda t}
\]

where:
\begin{itemize}
  \item $g_0^{(i)}$ is the initial sampled growth rate (FCFE or FCFF growth from the fundamental identity)
  \item $g_{\text{terminal}}$ is the long-run terminal growth rate (typically 2--3\%)
  \item $\lambda \in [0, 1]$ is the mean reversion speed parameter (typical value: 0.3)
  \item $t$ is the year index within the projection period
\end{itemize}

\textbf{Interpretation of $\lambda$:}
\begin{itemize}
  \item $\lambda = 0$: No reversion, constant growth $g_t^{(i)} = g_0^{(i)}$ for all $t$
  \item $\lambda = 0.3$: Moderate reversion (recommended), growth decays smoothly over 5--7 years
  \item $\lambda = 1.0$: Fast reversion, growth approaches terminal rate within 3--4 years
\end{itemize}

This time-varying formulation yields more conservative valuations than constant-growth models, as high initial growth rates gradually moderate to sustainable long-term levels. The feature is controlled by the \texttt{use\_time\_varying\_growth} configuration flag.

\paragraph{Cash Flow Computation}

Using the time-varying growth rates $g_t^{(i)}$, the projected cash flows are:

For FCFE:
\[
\text{FCFE}_t^{(i)} = \text{NI}_t^{(i)} + \text{D}_t^{(i)} - \text{CAPX}_t^{(i)} - \Delta \text{WC}_t^{(i)} + \text{NetBorrowing}_t^{(i)}
\]
where $\text{NetBorrowing}_t^{(i)} = \text{TDR} \cdot \text{Reinvestment}_t^{(i)}$

For FCFF:
\[
\text{FCFF}_t^{(i)} = \text{NOPAT}_t^{(i)} + \text{D}_t^{(i)} - \text{CAPX}_t^{(i)} - \Delta \text{WC}_t^{(i)}
\]
with $\text{NOPAT}_t^{(i)} = (1 - \text{CTR}) \cdot \text{EBIT}_t^{(i)}$

\paragraph{Terminal Value}

The terminal value uses a Gordon growth model:
\[
\text{TV}^{(i)} = \frac{\text{CashFlow}_h^{(i)} \cdot (1 + g_{\text{terminal}})}{r - g_{\text{terminal}}}
\]

where:
\begin{itemize}
  \item $r = \text{CE}$ for FCFE or $r = \text{WACC}$ for FCFF
  \item $g_{\text{terminal}}$ is the terminal growth rate (typically 2--3\%)
\end{itemize}

To avoid division by zero or unstable extrapolation, the denominator is lower-bounded:
\[
r - g_{\text{terminal}} \geq 0.01
\]

The terminal value is then discounted to present:
\[
\text{PV}_{\text{terminal}}^{(i)} = \frac{\text{TV}^{(i)}}{(1 + r)^h}
\]

\paragraph{Present Value}

The present value of cash flows is:
\[
\text{PV}_{\text{years}}^{(i)} = \sum_{t=1}^h \frac{\text{CashFlow}_t^{(i)}}{(1 + r)^t}
\]

Final present value of the firm:
\[
\text{PV}^{(i)} = \text{PV}_{\text{years}}^{(i)} + \text{PV}_{\text{terminal}}^{(i)}
\]

For FCFF, market value of debt per share is subtracted to get the per-share equity value:
\[
\text{IVPS}_{\text{FCFF}}^{(i)} = \frac{\text{PV}_{\text{FCFF}}^{(i)} - \text{MVB} + \text{CNOA}}{\text{SO}}
\]

\subsubsection{Statistical Analysis}

\paragraph{Simulation Matrix}

After $N$ simulations, we obtain an intrinsic value distribution for each asset and method (FCFE, FCFF):
\[
\{V^{(1)}, V^{(2)}, \ldots, V^{(N)}\}
\]

\paragraph{Kernel Density Estimation}

To estimate the probability density of intrinsic value, we apply a Gaussian kernel density estimator (KDE):
\[
\hat{f}(v) = \frac{1}{Nh} \sum_{i=1}^N K\left(\frac{v - V^{(i)}}{h}\right)
\]
where $K(\cdot)$ is the Gaussian kernel:
\[
K(u) = \frac{1}{\sqrt{2\pi}} e^{-\frac{u^2}{2}}
\]
and $h$ is the bandwidth selected using Scott's rule:
\[
h = N^{-1/5} \cdot \hat{\sigma}
\]
where $\hat{\sigma}$ is the sample standard deviation of $\{V^{(i)}\}$.

\paragraph{Value Surplus View}

Each simulation sample is transformed into a percentage surplus over market price:
\[
\text{Surplus}^{(i)} = \frac{V^{(i)} - P}{P}
\]

We then apply KDE on the transformed values to estimate the distribution of relative mispricing. This highlights the extent to which the simulated valuations exceed or fall short of the current price.

From the KDE and simulation samples, we extract:
\begin{itemize}
  \item $\mathbb{P}[\text{Surplus} > 0]$: probability the asset is undervalued
  \item $\mathbb{P}[\text{Surplus} < 0]$: probability it is overvalued
\end{itemize}

These quantities are calculated empirically via:
\[
\frac{1}{N} \sum_{i=1}^N \mathbb{I}[V^{(i)} > P]
\quad \text{and} \quad
\frac{1}{N} \sum_{i=1}^N \mathbb{I}[V^{(i)} < P]
\]

\subsubsection{Multi-Asset Portfolio Frontiers}

To explore portfolio-level valuation outcomes, we simulate many random portfolios over the set of assets.

\paragraph{Simulation Matrix}

Let $n$ be the number of assets and $S^{(j)}_i$ the $j$-th simulation of value surplus (as a percentage of market price) for asset $i$. The simulation matrix is:
\[
\mathbf{S} \in \mathbb{R}^{N \times n}, \quad \text{with } S_{ji} = \frac{V^{(j)}_i - P_i}{P_i}
\]

\paragraph{Random Portfolio Generation}

We generate $M$ random portfolio weight vectors $\mathbf{w}^{(k)} \in \mathbb{R}^n$ (typically $M = 5000$) using a Dirichlet distribution over the simplex:
\[
\mathbf{w}^{(k)} \sim \text{Dir}(\alpha \mathbf{1}), \quad \alpha = 1
\]

This ensures:
\[
\sum_{i=1}^n w^{(k)}_i = 1, \quad w^{(k)}_i \geq 0 \quad \forall i
\]

and provides uniform sampling over the weight simplex.

\paragraph{Portfolio Return Distribution}

For each portfolio $k$ and simulation $j$, the portfolio surplus is:
\[
R^{(k)}_j = \sum_{i=1}^n w^{(k)}_i S_{ji} = (\mathbf{w}^{(k)})^\top \mathbf{S}_j
\]

where $\mathbf{S}_j$ is the $j$-th row of the simulation matrix.

\paragraph{Portfolio Statistics}

For each portfolio $k$:

\begin{itemize}
  \item \textbf{Expected return} (mean surplus):
  \[
  \mu^{(k)} = \frac{1}{N} \sum_{j=1}^N R^{(k)}_j
  \]

  \item \textbf{Standard deviation} (valuation risk):
  \[
  \sigma^{(k)} = \sqrt{\frac{1}{N-1} \sum_{j=1}^N (R^{(k)}_j - \mu^{(k)})^2}
  \]

  Alternatively, using the covariance matrix:
  \[
  \sigma^{(k)} = \sqrt{(\mathbf{w}^{(k)})^\top \mathbf{\Sigma} \mathbf{w}^{(k)}}
  \]
  where $\mathbf{\Sigma}$ is the empirical covariance of $\mathbf{S}$.

  \item \textbf{Probability of loss}:
  \[
  p_{\text{loss}}^{(k)} = \frac{1}{N} \sum_{j=1}^N \mathbb{I}[R^{(k)}_j < 0]
  \]

  \item \textbf{Downside deviation} (Sortino denominator):
  \[
  \sigma_{\text{down}}^{(k)} = \sqrt{\frac{1}{N} \sum_{j=1}^N \min(0, R^{(k)}_j)^2}
  \]

  \item \textbf{Conditional Value-at-Risk} (CVaR at 95\%):
  Let $\alpha = 0.05$. Sort the portfolio returns: $R^{(k)}_{(1)} \leq R^{(k)}_{(2)} \leq \cdots \leq R^{(k)}_{(N)}$.

  The VaR is:
  \[
  \text{VaR}_{95}^{(k)} = R^{(k)}_{(\lfloor \alpha N \rfloor)}
  \]

  The CVaR is:
  \[
  \text{CVaR}_{95}^{(k)} = \frac{1}{\lfloor \alpha N \rfloor} \sum_{j=1}^{\lfloor \alpha N \rfloor} R^{(k)}_{(j)}
  \]

  \item \textbf{Maximum drawdown}:
  Compute cumulative returns over the simulation series and measure the largest peak-to-trough decline.
\end{itemize}

\paragraph{Efficient Frontier Plots}

Six distinct frontier views are produced, each optimizing a different risk-return trade-off:

\begin{enumerate}
  \item \textbf{Mean vs Standard Deviation} ($\mu$--$\sigma$):
  Mean-variance analysis. Identifies:
  \begin{itemize}
    \item Minimum variance portfolio: $\mathbf{w}_{\text{minvar}} = \arg\min_{\mathbf{w}} \mathbf{w}^\top \mathbf{\Sigma} \mathbf{w}$
    \item Maximum Sharpe portfolio: $\mathbf{w}_{\text{maxSharpe}} = \arg\max_{\mathbf{w}} \frac{\mu(\mathbf{w})}{\sigma(\mathbf{w})}$
  \end{itemize}

  \item \textbf{Mean vs Probability of Loss} ($\mu$--$p_{\text{loss}}$):
  Focuses on minimizing the chance of overvaluation. Identifies the portfolio with minimum $p_{\text{loss}}$ for a given expected return.

  \item \textbf{Mean vs Downside Deviation} ($\mu$--$\sigma_{\text{down}}$):
  Sortino-style frontier. Identifies the portfolio with maximum Sortino ratio:
  \[
  \text{Sortino}(\mathbf{w}) = \frac{\mu(\mathbf{w})}{\sigma_{\text{down}}(\mathbf{w})}
  \]

  \item \textbf{Mean vs CVaR} ($\mu$--$\text{CVaR}$):
  Tail risk frontier. Identifies the portfolio minimizing expected tail loss.

  \item \textbf{Mean vs VaR} ($\mu$--$\text{VaR}$):
  Similar to CVaR but focuses on the quantile rather than the conditional expectation.

  \item \textbf{Mean vs Maximum Drawdown} ($\mu$--$\text{MDD}$):
  Calmar-style frontier. Identifies the portfolio with maximum Calmar ratio:
  \[
  \text{Calmar}(\mathbf{w}) = \frac{\mu(\mathbf{w})}{\text{MDD}(\mathbf{w})}
  \]
\end{enumerate}

In each plot, the efficient frontier curve and key labeled portfolios (e.g., maximum return, minimum risk, optimal ratio) are annotated. Portfolio composition legends show the top 15 holdings for each optimal portfolio.

\subsubsection{Interpretation}

The probabilistic DCF framework yields a summary of valuation statistics and qualitative signals:

\begin{itemize}
  \item \textbf{Strong buy}: if both models (FCFE and FCFF) yield undervaluation ($\mathbb{P}[\text{Surplus} > 0] > 0.6$).
  \item \textbf{Buy}: if only one model suggests undervaluation.
  \item \textbf{Avoid}: if both suggest overvaluation ($\mathbb{P}[\text{Surplus} < 0] > 0.6$).
  \item \textbf{Hold}: if both suggest fair valuation.
\end{itemize}

The probabilistic approach provides several advantages over deterministic DCF:

\begin{enumerate}
  \item \textbf{Uncertainty quantification}: Rather than a single IVPS point estimate, we obtain a full distribution $\hat{f}(v)$, allowing probabilistic statements about valuation.

  \item \textbf{Confidence calibration}: The probability of undervaluation/overvaluation provides a confidence measure for investment decisions.

  \item \textbf{Portfolio-level analysis}: By combining simulation matrices across assets, we can construct efficient portfolios based on intrinsic value rather than historical prices.

  \item \textbf{Regime-invariant correlations}: The low inter-asset correlations (~0.0) observed in DCF-based returns reflect fundamental independence, unlike market-based returns which exhibit regime-dependent correlation structures.

  \item \textbf{Tail risk visibility}: The CVaR and maximum drawdown frontiers explicitly address worst-case scenarios, aligning with investor risk preferences during stress.
\end{enumerate}

The framework enables the construction of valuation-based portfolios that systematically exploit mispricing while managing fundamental uncertainty.
